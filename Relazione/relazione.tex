\documentclass[a4paper, 12pt]{report}
\usepackage[dvipsnames]{xcolor}
\usepackage{graphicx} % Required for inserting images
\usepackage{amsfonts}
\usepackage{amssymb}
\usepackage{tikz,lipsum,lmodern}
\usepackage[most]{tcolorbox}
\usepackage{pgfplots}
\usepackage{tabularx}
\usepackage{stmaryrd}
\usepackage{tikz} 
\usepackage{imakeidx}
\usepackage{blindtext}
\usepackage{hyperref}
\usepackage[rightcaption]{sidecap}
\usepackage{wrapfig}
\usepackage{cancel}

\hypersetup{
    colorlinks=true,
    linkcolor=black,
    filecolor=magenta,      
    urlcolor=blue,
    pdftitle={Relazione progetto e-commerce},
    pdfpagemode=FullScreen,
}

\setlength{\parindent}{0pt}
\setlength{\parskip}{5pt}


\makeindex[columns=3, title=Alphabetical Index, intoc]
\hypersetup{
    colorlinks=true,
    linkcolor=blue!50!green,
    filecolor=magenta,      
    urlcolor=cyan,
    pdftitle={Overleaf Example},
    pdfpagemode=FullScreen,
    }

\newcommand\scalemath[2]{\scalebox{#1}{\mbox{\ensuremath{\displaystyle #2}}}}
\definecolor{bananamania}{rgb}{0.98, 0.91, 0.71}
\definecolor{amaranth}{rgb}{0.9, 0.17, 0.31}
\definecolor{amethyst}{rgb}{0.6, 0.4, 0.8}
\definecolor{darktangerine}{rgb}{1.0, 0.66, 0.07}
\definecolor{cerise}{rgb}{0.87, 0.19, 0.39}
\definecolor{babyblue}{rgb}{0.54, 0.81, 0.94}


\title{Relazione su \\
Controllo formazione droni}
\author{P. Bettoni (1998044),\\ A. Coppola (2003964),\\S. Di Cesare (1938649)}
\date{\today}

\begin{document}
\maketitle
\newpage
\tableofcontents
\newpage
\chapter*{Descrizione Generale}
Si vuole progettare un sistema di controllo di n droni che sorvegli una data area chiusa.

Il sistema gestisce una torre di controllo e ricarica, dalla quale i droni devono partire e devono tornare per ricaricarsi. 
I droni una volta connessi alla torre, seguono le sue indicazioni e si dirigono verso ogni punto gli venga assegnato dalla torre.
I droni seguono un sistema di volo "a tappe", in cui la torre invia loro una posizione indicata con delle coordinate cartesiane
relative alla dimensione dell'area da sorvegliare. Una volta che il drone arriva alla posizione ricevuta lo comunica alla torre che risponde con la posizione successiva, delineando così un percorso.
Quando il drone è scarico lo comunica alla torre e si direziona alla torre di controllo, per essere ricaricato.

Sta alla torre quindi il compito di calcolare, mentre i droni sono in volo, il percorso migliore per far sì che ogni punto dell'area venga sorvegliato 
il più frequentemente possibile, tenendo conto dei punti visitati, dei droni in volo, e dei punti che visitano prima di scaricarsi.

\section*{Idea dell'algoritmo}
Contando di avere un database che contiene per ogni drone:
\begin{enumerate}
    \item posizione
    \item stato
\end{enumerate}

un drone può trovarsi in solo $n$ stati:
\begin{enumerate}
    \item \textbf{Ready}: il drone è carico e si trova alla posizione di 
\end{enumerate}



\end{document}